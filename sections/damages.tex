\section{Damages}

\subsection{Compensatory Damages}
 
\subsubsection{Loss of Enjoyment and Pain and Suffering: \emph{McDougald v.  
Garber}}

\begin{enumerate}
    \item Nonpecuniary damages compensate a victim for physical and emotional 
    consequences, such as pain and suffering or loss of ability. Pecuniary 
    damages compensate for economic loss.
    \item The victim here suffered permanent brain damage and entered a coma 
    after a C-section.
    \item After a remittitur, the plaintiff won \$2,000,000 for conscious pain 
    and suffering and loss of the pleasures and pursuits of life. Her husband 
    won \$1,500,000 for loss of services.
    \item The trial court accepted the plaintiffs' argument that damages for 
    loss of enjoyment of life could be awarded even though the plaintiff was not 
    aware of the loss. The appellate court held (1) that awareness is required 
    to recover for loss of enjoyment of life and (2) separating damages for pain 
    and suffering from damages for loss of enjoyment is not possible because 
    suffering and enjoyment cannot be directly converted into monetary values.
    \item The purpose of tort damages is to compensate the victim. They should 
    not be punitive unless the defendant acted with malice.\footnote{Casebook p. 
    607.}
    \item Pain and suffering generally encompass loss of 
    enjoyment.\footnote{Casebook p. 610.}
    \item Judge Titone, dissenting: Pain and suffering are logically distinct 
    from loss of enjoyment. Damages for each should be kept separate.
\end{enumerate}

\subsubsection{Collateral Source Rule: \emph{Helfend v. Southern California 
Rapid Transit District}}

\begin{enumerate}
    \item A bus crushed the plaintiff's arm.
    \item At trial, the defendant sought to introduce evidence showing that 
    insurance had paid 80\%, possibly more, of the plaintiff's medical bills.  
    The court ruled that the defendants could not show that the plaintiff had 
    received medical coverage from any \textbf{collateral source}.
    \item On appeal, the defendant argued that the trial court erred in 
    preventing evidence that a collateral source had paid the plaintiff's 
    medical bills and denying the defendant the opportunity to discover whether 
    the defendant had recovered costs from more than one collateral source.
    \item The collateral source rule exists to create an incentive for people to 
    buy health insurance. Moreover, attorneys generally draw compensation from 
    damages at trial, which would be put in peril if juries knew the plaintiff 
    had already recovered from an insurance company.
    \item Changes to the collateral source rule ``would be more effectively 
    accomplished through legislative reform.''\footnote{Casebook p. 614. The 
    California legislature did in fact remove the collateral source rule---see 
    \emph{Fein}, p. 629.}
    \item Subrogation clauses in insurance policies entitle the insurer to tort 
    damages up to the amount the insurer paid.
\end{enumerate}

\subsection{Punitive Damages}

\subsubsection{\emph{State Farm Mutual Automobile Ins. Co. v. Campbell}}

\begin{enumerate}
    \item The plaintiff, Campbell, tried to pass six cars on the highway. To 
    avoid colliding with the plaintiff, Ospital swerved, killing himself and 
    permanently disabling Slusher.
    \item Campbell initially insisted he was not at fault, but a consensus 
    emerged that his attempted pass caused the accident. But State Farm 
    contested liability and declined settlement offers from Slusher and 
    Ospital's estate. State Farm assured the Campbells that their assets were 
    safe, they had no liability, and they did not need to seek outside counsel. 
    When the jury determined that Campbell was at fault and awarded \$185,849 in 
    damages, State Farm told the Campbells ``to put for sale signs on your 
    property to get things moving.''\footnote{Casebook p. 615.}
    \item While the appeal was pending, the Campbells struck a deal with Ospital 
    and Slusher to pursue a bad faith action State Farm. Slusher and Ospitals 
    attorneys would represent the three of them, with Slusher and Ospital 
    entitled to 90\% of any verdict against State Farm.
    \item The jury awarded \$145 million in punitive damages and \$1 million in 
    compensatory damages against State Farm.
    \item The Supreme Court here considered whether the punitive damages were 
    excessive. It noted that compensatory damages redress concrete losses while 
    \textbf{punitive damages aim for deterrence and retribution}.
    Supreme Court here considered whether the punitive damages were excessive 
    and in violation of due process.
    \item In \emph{BMW v. Gore}, the Supreme Court established three criteria 
    for reviewing punitive damages:\footnote{Casebook p. 616.}
    \begin{enumerate}
        \item The reprehensibility of the defendant's conduct.
        \item The disparity between actual/potential harm and the punitive 
        damages.
        \item The difference between damages awarded and damages in similar 
        civil cases.
    \end{enumerate}
    \item Applying the \emph{Gore} criteria, the court found that ``this case is 
    neither close nor difficult. It was error to reinstate the jury's \$145 
    million punitive damages award.''\footnote{Casebook p. 617.}
    \item Four states have abolished punitive damages (Levy).
    \item Punitive damages resemble criminal punishment in some respects
    (e.g., retribution). The burden of proof is lower (preponderance of the
    evidence vs. beyond a reasonable doubt), but the punishments are purely
    economic.
\end{enumerate}

\subsubsection{Cal. Civ. Code \S\ 3294}

\begin{enumerate}
    \item California allows recovery of punitive damages for oppression, 
    fraud, malice, and homicide resulting from a felony.
    \item Burden of proof for punitive damages is ``clear and convincing
    evidence.''\footnote{Reader p. 102.}
\end{enumerate}
