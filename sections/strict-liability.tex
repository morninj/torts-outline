\section{Strict Liability}

\subsection{Traditional Strict Liability}

\begin{enumerate}
    \item Strict liability means that the plaintiff's prima facie case does not 
    need to prove that the defendant acted in a blameworthy fashion.  \item 
    Plaintiff has to prove other elements: cause-in-fact, proximate cause.
    \item Most important areas: legislative programs (e.g., workers' comp.---not 
    usually referred to as strict liability, but that's how it operates, because 
    fault is generally not an issue), and certain ``abnormally dangerous'' 
    activities (e.g., keeping wild animals, blasting, use of poisons), strict 
    products liability.
    \item An anomally in the field of torts historically---but liability without 
    fault is present in other areas of the law: contracts (breach of contract 
    generally does not involve fault).
    \item Premised on the need for greater loss distribution that what would 
    occur under negligence.
\end{enumerate}

% Abnormally dangerous activities: Restatement Second of Torts § 520
%     * On the one hand, most states don't require proof of all factors. "Not a 
%     matter of common usage," however, is almost always required.
% 
% "(d) extent to which the activity is not a matter of common usage;"---without 
% this clause, auto accidents would likely be strict liability crimes.
% 
% Yukon Equipment, Inc. v. Fireman's Fund Ins. Co.
% 
% p. 63 priebe kennard dissent: -- civ code 3342 (dog bites): in CA, strict 
% liability for dog owners (except for veterinarians).

\subsubsection{\emph{Siegler v. Kuhlman}}

\begin{enumerate}
    \item Overturned gas trailer caused fire.
    \item Transporting gas by truck is abnormally dangerous. It possesses all of 
    the Restatement factors for strict liability.
    \item Trial court: defendants overcame charges of negligence.
    \item Holding: reversed and remanded for retrial on strict liability.
\end{enumerate}

\subsubsection{\emph{Indiana Harbor Belt Railroad Co. v. American Cyanamid Co.}}

\begin{enumerate}
    \item Cyanamid loaded 20k gallons of acrylonitrile in a leased railroad car.
    \item Indiana Harbor Belt asserted (1) negligent maintenance of the train 
    car and (2) strict liability because transport of bulk acrylonitrile through 
    Chicago is abnormally dangerous.
    \item Distinction from \emph{Siegler}: defendant there was the transporter; 
    here it is the shipper.
    \item Harm here was the result of carelessness, not inherent danger.  
    Negligence would have been an effective deterrent.
    \item Reversed and remanded to be tried on negligence.
\end{enumerate}

\subsubsection{\emph{Kelley v. R.G. Industries, Inc.}}

\begin{enumerate}
    \item Gunshot victim claimed the manufacturing and marketing of handguns is 
    abnormally dangerous. Court rejected the argument because under SL, the 
    activity must be dangerous in relation to the area where it occurs.
\end{enumerate}

\subsubsection{\emph{Foster v. Preston Mill Co.}}

\begin{enumerate}
    \item Blasting operations caused a mink to kill its kittens. Court rejected 
    the strict liability argument because liability only exists for harms 
    resulting ``from that which makes the activity ultrahazardous.''
\end{enumerate}

\subsection{Products Liability}

\begin{enumerate}
    \item % TODO
    \item A manufacturer (or anyone in the chain of distribution) is strictly 
    liable when an article is placed on the market knowing that it is to be used 
    without further inspection for defects and proves to have a defect that 
    causes injury.
    \item Strict liability is not based on fault. The rationale is loss 
    distribution. % TODO other rationales
    \item In addition to strict liability, manufacturers can also be held liable 
    for negligence, express/implied warranty, and representation.
    \item Defect is defined as: % TODO
    \item Issues in defining defect have split jurisdictions
    \begin{enumerate}
        \item Must the product be ``unreasonably dangerous''? The majority of 
        states require it, but CA says no, because it introduces a negligence 
        standard. Can there be liability for an open an obvious defect? CA no, 
        others yes.
        \item Must the product be in a condition not anticipated by the buyer, 
        i.e., beyond the consumer's expectation? Some jurisdictions make this a 
        requirement. In CA, that's one method of getting to strict liability 
        (\emph{Barker}), but it's not necessary. Some say this introduced 
        negligence, but (1) it's in hindsight, not foresight and (2) the burden 
        of proof is on the defendant, not the plaintiff. But see \emph{Sewell}: 
        if the product is too complex, the consumer expectations test does not 
        apply. % TODO: add the barker two-prong test.
    \end{enumerate}
    \item No strict liability for prescription pharmaceuticals.
    \item State-of-the-art defense: is the defendant relieved of liability if at 
    the time of the manufacture, nobody could have made it more safe. Some 
    states have adopted this rule; in CA it's only adopted in a few 
    areas---pharmaceuticals, warning defects (\emph{Anderson}).
    \item Manufacturing defect: the product is different than all the others 
    produced.
    \item Warning defect: inadequate labels or instructions. Purposes: inform 
    the consumer of dangers to let her avoid buying it or to use it more safely.
    \item Restatement (Third) of torts would radically shift products liability 
    in favor of manufacturers. Plaintiff would have to prove the existence of a 
    reasonable alternative design. The \emph{Potter} court rejected the rule as 
    unduly requiring plaintiffs to retain expert witnesses. No chance that this 
    rule will be adopted in CA in the foreseeable future.
    \item Restatement (Third) also tries to combine products liability into a 
    single principle (\S\ 550.1). Levy fears this would wipe away much of 
    existing products liability law.
    \item Strict liability does not allow recovery for economic damages.
    \item If plaintiff is negligent, we will apply comparative negligence, even 
    in strict products liability cases. There's dispute about whether fault can 
    logically apply in strict liability contexts.
    \item Preemption: when do federal rules preempt state law? There are three 
    types of preemption under the Supremacy Clause: express, conflict, field.
    \item Can there be liability for component parts of a product? Cases are not 
    all in agreement. Best rule: (1) if the component part is defective, there 
    can be liability. (2) If the component parts manufacturer was intimately 
    involved in the design of the whole product, it can be held liable for the 
    whole product.
    \item The ``sophisticated/professional user'' defense: a manufacturer 
    generally owes no duty to warn professionals against the danger if the 
    danger is generally known to the profession.
    \item \emph{Daly}, comparative negligence: can you apply assumption of risk 
    to a strict products liability case? The answer will likely be yes, though 
    there is no California Supreme Court case that directly addresses it.
    \item \end{enumerate}

\subsubsection{\emph{Pillars v. R. J. Reynolds Tobacco Co.}}

\begin{enumerate}
    \item Human toe in chewing tobacco triggered strict liability.
\end{enumerate}

\subsubsection{Strict products liability: \emph{Greenman v. Yuba Power Products, 
Inc.}}

\begin{enumerate}
    \item This is the first case to find strict products liability for defective 
    products.
    \item A piece of wood flew out of a woodworking tool, the Shopsmith, 
    injuring the plaintiff, Greenman.
    \item 10.5 months later, he sued the manufacturer, Yuba, and the retailer 
    for breach of warranty and negligence.
    \item The court found that (1) the retailer was not negligent and did not 
    breach an express warranty, and (2) the manufacturer did not breach an 
    implied warranty. Thus, the only valid causes of action were (1) a breach of 
    implied warranty against the retailed and (2) negligence and a breach of 
    express warranty against the manufacturer. The jury found for the retailer 
    and found \$65,000 against Yuba.
    \item Yuba appealed; Greenman sought appeal against the retailer only if the 
    judgment against Yuba was reversed.
    \item The jury could have reasonably found that Yuba negligently 
    manufactured the Shopsmith.\footnote{Casebook p. 520.}
    \item The requirement that consumers need not give notice of injury to 
    manufacturers with whom they have not directly dealt. Thus, the plaintiff's 
    cause of action was not barred.
    \item The manufacturer can be held strictly liable for a defective product 
    even in the absence of an express warranty: \textbf{``A manufacturer is 
    strictly liable in tort when an article he places on the market, knowing 
    that it is to be used without inspection for defects, proves to have a 
    defect that causes injury to a human being.''}\footnote{Casebook p. 521.}
    \item Liability for defective products is governed by strict liability, not 
    contract warranties.  \item The purpose of strict liability for defective 
    products is to ensure that manufacturers bear the costs of injuries to 
    consumers.
    \item \textbf{Warranties}:
    \begin{enumerate}
        \item \emph{Express warranties}: created when the seller makes factual 
        assertions about a product.
        \item \emph{Implied warranties}: (1) ``implied warranty of 
        merchantability'' is a guarantee that products conform to their 
        description and are safe for their intended use; (2) ``implied warranty 
        of fitness for a particular purpose'' is created when the seller has 
        reason to know that the buyer buys the goods for a particular purpose. 
        \footnote{Casebook p. 523.}
        \item The advantage of basing a products liability case on a warranty 
        theory is that liability is strict and there can be compensate for pure 
        economic loss. The disadvantage is that sellers can limit remedies or 
        disclaim warranties altogether. Warranties also historically require 
        prompt notice of dissatisfaction to the defendant.\footnote{Casebook p. 
        524.}
    \end{enumerate}
    \item \textbf{Misrepresentation}: another theory for product liability (in 
    addition to negligence and warranty theory). It holds manufacturers liable 
    for harm caused by justified reliance on the misrepresentation.
    \item \textbf{Strict product liability}: a fourth theory. It imposes 
    liability on manufacturers for defective products that proximately cause 
    personal and property injuries. [What about economic injuries?] This is the 
    theory in \emph{Greenman}.
\end{enumerate}

\subsubsection{\emph{Lee v. Crookston Coca-Cola Bottling Co.}}

\begin{enumerate}
    \item Coke bottle exploded in waitress's hands.
    \item Four policy justifications for strict products liability:
    \begin{enumerate}
        \item Discourage marketing of defective products.
        \item Put burden of loss on manufacturer.
        \item Maximize legal protections for consumers.
        \item Allow injured parties to bring actions directly against those who 
        caused the injuries without involving others in the distribution chain.
    \end{enumerate}
\end{enumerate}

\subsubsection{\emph{Gray v. Manitowoc Company}}

\begin{enumerate}
    \item Crane hit construction worker, who argued that mirrors should have 
    been provided. Court found that the safety hazards of this type of crane 
    were well known in the industry and thus was not ``dangerous to a degree not 
    anticipated by the ordinary consumer of this product.''\footnote{Casebook p. 
    531.}
\end{enumerate}

\subsubsection{\emph{Roysdon v. R.J. Reynolds Tobacco Co.}}

\begin{enumerate}
    \item % TODO
\end{enumerate}

\subsubsection{\emph{Barker v. Lull Engineering Co., Inc.}}

\begin{enumerate}
    \item % TODO
\end{enumerate}

\subsubsection{State-of-the-art defense: \emph{Beshasda v. Johns-Manville 
Products Corp.}}

\begin{enumerate}
    \item This was a consolidated case against six asbestos manufacturers. The 
    defendants' ``state-of-the-art'' defense argued that the dangers of asbestos 
    were unknowable at the time the injuries in question occurred.
    \item The trial court denied the plaintiffs' motion to strike the 
    state-of-the-art defense.
    \item The plaintiffs claimed strict liability for failure to warn. ``The 
    issue is whether the medical community's presumed unawareness of the dangers 
    of asbestos is a defense to plaintiffs' claims.''\footnote{Casebook p. 549.}
    \item The court distinguished negligence, which is conduct oriented, from 
    strict liability, which is product oriented.
    \item There is a two-part \textbf{risk equity} test to determine whether a 
    product is safe:]\
    \begin{enumerate}
        \item Does its utility outweigh its risk?
        \item Has that risk been reduced to the greatest extent possible 
        consistent with the product's utility?\footnote{Casebook p. 551.}
    \end{enumerate}
    \item In strict liability cases, there is no need to prove that the 
    manufacturer knew or should have known of the product's danger. Knowledge is 
    imputed to the manufacturer. ``...in strict liability cases, culpability is 
    irrelevant.''\footnote{Casebook p. 552.} The state-of-the-art defense is a 
    negligence defense because it rests on the defendant's conduct.
    \item There are three reasons for imposing strict liability for failure to 
    warn:
    \begin{enumerate}
        \item \emph{Risk spreading}: spreading costs of harm to manufacturers 
        and purchasers is preferable to imposing it on innocent consumers.
        \item \emph{Accident avoidance}: industries play an important role in 
        safety research, and we want them to maximize it.
        \item \emph{Fact finding}: the dangers of asbestos \emph{could have been 
        known}, but weren't. Regardless, it's better to leave out the negligence 
        concept of knowability, because the framework here is strict liability, 
        not negligence.
    \end{enumerate}
    \item The court granted the plaintiffs' motion to strike the 
    state-of-the-art defense.
    \item (In contrast to New Jersey, the majority trend is to allow the 
    state-of-the-art defense, including in California.\footnote{Casebook p.  
    557.})
\end{enumerate}

\subsubsection{Federal preemption: \emph{Riegel v. Medtronic, Inc.}}

\begin{enumerate}
    \item The Medical Device Amendments (MDA) to the FDCA established various 
    levels of federal oversight for medical devices depending on their risks.  
    Devices that were already on the market were grandfathered, and new devices 
    that were ``substantially similar'' to the existing devices could also 
    sidestep premarket approval.
    \item Here, the doctor inflated a balloon catheter beyond the pressure limit 
    indicated on its label, causing injury to the plaintiff.
    \item The district court held (1) that the MDA preempted the plaintiff's 
    common law tort claims and (2) that the MDA preempted the plaintiff's 
    negligent manufacturing claim because it did not claim that the manufacturer 
    violated federal law.\footnote{Casebook p. 562.}
    \item Justice Scalia:
    \begin{enumerate}
        \item The MDA only preempts state requirements that are different or in 
        addition to the applicable federal requirements. The court here, relying 
        on \emph{Lohr}, found that state law negligence and strict liability 
        claims are different and therefore the MDA preempts them.
        \item If federal regulations did not preempt state common law, then 
        states and individual juries would be able to undermine the FDA's expert 
        evaluations and policies.
        \item The consequence is that the FDA's approval of the device preempts 
        state tort law actions based on negligence and strict 
        liability.\footnote{Casebook p. 565.}
    \end{enumerate}
\end{enumerate}

\subsubsection{\emph{McKenney v. PurePac Pharmaceutical}}

\begin{enumerate}
    \item PurePac manufactured the generic drug metoclopramide. McKenney claimed 
    she was injured because of ``false or misleading statements'' in the drug's 
    labeling.\footnote{Casebook p. 89.}
    \item The CA Superior Court sustained PurePac's demurrer and entered summary 
    judgment in its favor.
    \item In its demurrer, PurePac contended that McKenney's claim was barred by 
    the defense of federal preemption. Because it submitted its labeling to the 
    FDA and won approval, PurePac argued it could not be held liable for state 
    tort law claims regarding any deficiencies in the labeling.
    \item \emph{Brown} and \emph{Carlin} affirmed strict tort liability for 
    pharmaceutical manufacturers in California.
    \item The court found that FDA approval of labeling does not preempt state 
    tort claims against manufacturers.
    \item Reversed (demurrer rejected).
    \item Reconciling \emph{Riegel} and \emph{McKenney}: courts will likely 
    allow state causes of action that are parallel to the federal rules.
\end{enumerate}


\subsubsection{Restatement (Third) Approach: \emph{Potter v. Chicago Pneumatic 
Tool Co.}}

\begin{enumerate}
    \item Plaintiffs claim they were injured from excessive vibrations as a 
    result of defective warnings on the defendant's product.
    \item Courts are divided on the definition of design defects.
    \item The Restatement (Third) requires plaintiff to prove the existence of a 
    ``reasonable alternative design.''\footnote{Casebook p. 566.} The defendants 
    argue that the court should adopt this standard.
    \item The court here reasoned that the Restatement (Third) approach puts an 
    undue burden on plaintiffs by requiring expert witnesses even in cases where 
    a lay jury could infer a design defect. Moreover, cases exist where a 
    product is defective even though no alternative design exists.
    \item The Restatement (Third) holds that a product is defective only if 
    there are foreseeable risks that a reasonable alternative design would have 
    avoided. Thus, it allows the state-of-the-art defense and imposes a burden 
    on plaintiffs more onerous than ordinary negligence (because under ordinary 
    negligence, the plaintiff only needs to prove a foreseeable risk, but not 
    the existence of an alternative design).
\end{enumerate}

% \subsubsection{Economic Damages: \emph{Two Rivers Company v. Curtiss Breeding 
% Service}}
% 
% \begin{enumerate}
%     \item % TODO
% \end{enumerate}
% 
% \subsubsection{\emph{Daly v. General Motors Corp.}}
% 
% \begin{enumerate}
%     \item % TODO
% \end{enumerate}
% 
% \subsubsection{\emph{Gonzales v. Autoliv}}
% 
% \begin{enumerate}
%     \item % TODO
% \end{enumerate}
% 
\subsection{Compensatory Damages}
 
\subsubsection{Loss of Enjoyment and Pain and Suffering: \emph{MacDougald v.  
Garber}}

\begin{enumerate}
    \item Nonpecuniary damages compensate a victim for physical and emotional 
    consequences, such as pain and suffering or loss of ability. Pecuniary 
    damages compensate for economic loss.
    \item The victim here suffered permanent brain damage and entered a coma 
    after a C-section.
    \item After a remittitur, the plaintiff won \$2,000,000 for conscious pain 
    and suffering and loss of the pleasures and pursuits of life. Her husband 
    won \$1,500,000 for loss of services.
    \item The trial court accepted the plaintiffs' argument that damages for 
    loss of enjoyment of life could be awarded even though the plaintiff was not 
    aware of the loss. The appellate court held (1) that awareness is required 
    to recover for loss of enjoyment of life and (2) separating damages for pain 
    and suffering from damages for loss of enjoyment is not possible because 
    suffering and enjoyment cannot be directly converted into monetary values.
    \item The purpose of tort damages is to compensate the victim. They should 
    not be punitive unless the defendant acted with malice.\footnote{Casebook p. 
    607.}
    \item Pain and suffering generally encompass loss of 
    enjoyment.\footnote{Casebook p. 610.}
    \item Judge Titone, dissenting: Pain and suffering are logically distinct 
    from loss of enjoyment. Damages for each should be kept separate.
\end{enumerate}

\subsubsection{Collateral Source Rule: \emph{Helfend v. Southern California 
Rapid Transit District}}

\begin{enumerate}
    \item A bus crushed the plaintiff's arm.
    \item At trial, the defendant sought to introduce evidence showing that 
    insurance had paid 80\%, possibly more, of the plaintiff's medical bills.  
    The court ruled that the defendants could not show that the plaintiff had 
    received medical coverage from any \textbf{collateral source}.
    \item On appeal, the defendant argued that the trial court erred in 
    preventing evidence that a collateral source had paid the plaintiff's 
    medical bills and denying the defendant the opportunity to discover whether 
    the defendant had recovered costs from more than one collateral source.
    \item The collateral source rule exists to create an incentive for people to 
    buy health insurance. Moreover, attorneys generally draw compensation from 
    damages at trial, which would be put in peril if juries knew the plaintiff 
    had already recovered from an insurance company.
    \item Changes to the collateral source rule ``would be more effectively 
    accomplished through legislative reform.''\footnote{Casebook p. 614. The 
    California legislature did in fact remove the collateral source rule---see 
    \emph{Fein}, p. 629.}
    \item Subrogation clauses in insurance policies entitle the insurer to tort 
    damages up to the amount the insurer paid.
\end{enumerate}

\subsection{Punitive Damages}

\subsubsection{\emph{State Farm Mutual Automobile Ins. Co. v. Campbell}}

\begin{enumerate}
    \item The plaintiff, Campbell, tried to pass six cars on the highway. To 
    avoid colliding with the plaintiff, Ospital swerved, killing himself and 
    permanently disabling Slusher.
    \item Campbell initially insisted he was not at fault, but a consensus 
    emerged that his attempted pass caused the accident. But State Farm 
    contested liability and declined settlement offers from Slusher and 
    Ospital's estate. State Farm assured the Campbells that their assets were 
    safe, they had no liability, and they did not need to seek outside counsel. 
    When the jury determined that Campbell was at fault and awarded \$185,849 in 
    damages, State Farm told the Campbells ``to put for sale signs on your 
    property to get things moving.''\footnote{Casebook p. 615.}
    \item While the appeal was pending, the Campbells struck a deal with Ospital 
    and Slusher to pursue a bad faith action State Farm. Slusher and Ospitals 
    attorneys would represent the three of them, with Slusher and Ospital 
    entitled to 90\% of any verdict against State Farm.
    \item The jury awarded \$145 million in punitive damages and \$1 million in 
    compensatory damages against State Farm.
    \item The Supreme Court here considered whether the punitive damages were 
    excessive. It noted that compensatory damages redress concrete losses while 
    \textbf{punitive damages aim for deterrence and retribution}.
    Supreme Court here considered whether the punitive damages were excessive 
    and in violation of due process.
    \item In \emph{BMW v. Gore}, the Supreme Court established three criteria 
    for reviewing punitive damages:\footnote{Casebook p. 616.}
    \begin{enumerate}
        \item The reprehensibility of the defendant's conduct.
        \item The disparity between actual/potential harm and the punitive 
        damages.
        \item The difference between damages awarded and damages in similar 
        civil cases.
    \end{enumerate}
    \item Applying the \emph{Gore} criteria, the court found that ``this case is 
    neither close nor difficult. It was error to reinstate the jury's \$145 
    million punitive damages award.''\footnote{Casebook p. 617.}
    \item Four states have abolished punitive damages (Levy).
    \item Punitive damages resemble criminal punishment in some respects
    (e.g., retribution). The burden of proof is lower (preponderance of the
    evidence vs. beyond a reasonable doubt), but the punishments are purely
    economic.
\end{enumerate}

\subsubsection{Cal. Civ. Code \S\ 3294}

\begin{enumerate}
% TODO reader     \item 
    \item Burden of proof for punitive damages is ``clear and convincing
    evidence.''\footnote{Reader p. 102.}
\end{enumerate}

\subsection{Vicarious Liability}

\begin{enumerate}
    \item Vicarious liability is the doctrine that holds one party liable
    because of that party's relationship to a tortfeasor.
    \item The most common form of vicarious liability is respondeat superior,
    which hold an employer liable for its employees' torts.
    \item You cannot insure for punitive damages.
\end{enumerate}

\subsubsection{Respondeat Superior: \emph{Rodgers v. Kemper Construction Co.}}

\begin{enumerate}
    \item Respondeat superior: an employer is liable for an employee's torts 
    committed within the scope of employment.
    \item Kemper employees frequently drank after their shifts. Herd and O'Brien 
    finished their shifts, drank a few beers, walked across the job site, and 
    asked Rodgers for a ride on the bulldozer. They beat him up when he refused. 
    Rodgers later asked Kelley to help find out his assailants' identities. As 
    Rodgers wrote down O'Brien's license plate number, Herd, O'Brien, and a 
    third, Dieffenbauch, attacked Rodgers and Kelley, causing serious injury to 
    both.
    \item The trial court found Kemper liable for the injuries under respondeat 
    superior.
    \item On appeal, Kemper argued that it could not be held liable under 
    respondeat superior because (1) the assault occurred after O'Brien and Herd 
    had finished their shift and (2) the assault was based on personal malice.
    \item The appellate court rejected both of Kemper's arguments on the grounds 
    that the injuries resulted from ``a dispute arising out of the 
    employment.''\footnote{Casebook p. 636.}
\end{enumerate}

\subsubsection{Going-and-Coming Rule: \emph{Caldwell v. A.R.B., Inc.}}

\begin{enumerate}
    \item Generally, there is no employer liability for an employee's
    actions while commuting.
    \item A.R.B. workers at a Shell Oil plant were sent home because of bad 
    weather. Brandon offered to give Richardson a ride home. On the way, Brandon 
    collided with Caldwell. Caldwell sued Brandon and A.R.B., alleging that 
    Brandon was acting within the scope of employment. A.R.B. filed a motion for 
    summary judgment on the grounds that Brandon was outside of the scope of 
    employment, which the trial court granted.
    \item Although A.R.B. compensated its employees for travel expenses, the 
    appellate court held that Brandon was outside the scope of employment.
\end{enumerate}

\subsubsection{Independent Contractors: \emph{Mavrikidis v. Petullo}}

\begin{enumerate}
    \item Gerald Petullo was driving a dump truck full of hot asphalt when he 
    collided with Mavrikidis. Petullo and his father had been working on 
    renovations of the Clar Pine service station, which Karl Pascarello owned,
    \item Malvrikidis argued that Petullo was an employee of Pascarello, but the 
    court held that he was an independent contractor. Restatement (Second) of 
    Agency lists several factors for determining whether an actor is an 
    employee.\footnote{Casebook p. 647.} Applying these factors, the court found 
    that Petullo was not an employee.
    \item The court listed three exceptions to independent contractor 
    non-liability: (1) when the principal retains control of the work, (2) when 
    the principal hires an incompetent contractor, and (3) when the work is 
    inherently dangerous. None of these exceptions applied to Petullo.
    \item Pascarello was not liable.
\end{enumerate}

\subsubsection{Vicarious Liability for Children: \emph{Wells v. Hickman}}

\begin{enumerate}
    \item L.H. (15) beat D.E. (12) to death. D.E.'s mother (Wells) filed a 
    wrongful death action against L.H.'s mother (Hickman) and grandparents.
    \item An Indiana statute held parents strictly liable for their children's 
    knowing, intentional, or reckless torts for damages up to \$3,000. The trial 
    court granted summary judgment for Hickman.
    \item The appellate court here reasoned that there are four common law 
    exceptions to the rule that a parent is not liable for a child's torts: (1) 
    when the parent entrusts the child with an instrumentality that may pose 
    danger to others, (2) where the child acts as the parents' agent, (3) where 
    the parent consents, and (4) where the parent fails to exercise control when 
    the parent knows or should know that injury is possible.
    \item Wells argued that the Hickman's actions fell under the fourth 
    exception and that the statute did not limit recovery. The appellate court 
    agreed.
\end{enumerate}
