\section{Defenses to Intentional Torts}

\subsection{Self Defense}

Force intended to inflict death or serious injury is only reasonable in response to the immediate threat of serious bodily injury or death.

The Restatement of Torts also requires retreat if safely possible (except from the victim's own dwelling) before the victim can respond with force intended to inflict serious bodily injury or death. Diamond notes that most courts disagree.\footnote{Casebook p. 64.}

\begin{enumerate}
    \item \emph{Drabek v. Sabley}: Ten-year-old Drabek and friends were throwing snowballs at passing cars. One driver, Sabley, stopped, caught Drabek, took him by the arm to his car, and drove him back to the village of Williams Bay. He turned Drabek over to the police. Drabek was with Sabley for a total of 15-20 minutes. The court held that Sabley was justified in preventing the commission of a crime, and so it was reasonable to admonish Drabek and march him home. But it was not reasonable to detain him and take him to the police station, so Sabley is liable for false imprisonment and nominal battery. Remanded to determine compensatory (but not punitive) damages.
\end{enumerate}

\subsection{Private Necessity}

``Private necessity is a privilege which allows the defendant to interfere with the property interests of an innocent party in an effort to avoid a greater injury. The privilege is incomplete since the actor must still compensate the victim for the property.''\footnote{Casebook pp. 69--70.}

\begin{enumerate}
    \item \emph{Vincent v. Lake Erie Transp. Co}: Defendant was moored at the plaintiff's dock to unload goods when a severe storm struck. Defendant kept his boat secured (and repeatedly replaced damaged or broken lines) to the dock throughout the storm, causing \$500 in damages to the dock. The court held that an actor is justified in using another's property in extreme circumstances, but will be held responsible for any damages incurred.
\end{enumerate}
