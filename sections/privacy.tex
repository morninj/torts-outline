\section{Privacy}

\begin{enumerate}
    \item There are traditionally four privacy torts:
    \begin{enumerate}
        \item Intrusion upon seclusion.
        \item Unauthorized use of name or likeness.
        \item Giving unreasonable publicity to private matters.
        \item Publicly characterizing a party in a false light.
    \end{enumerate}
\end{enumerate}

\subsection{Intrusion upon Seclusion}

\begin{enumerate}
    \item The intrusion must be highly offensive to a reasonable person.
    \item There is no requirement of publication or communication.
\end{enumerate}

\subsubsection{\emph{Pearson v. Dodd}}

\begin{enumerate}
    \item Members of Senator Thomas Dodd's staff surreptitiously copied 
    documents from the Senator's office and sent them to reporters, Pearson 
    and Anderson.
    \item The distric court denied summary judgment for invasion of privacy.
    \item The Fifth Circuit affirmed, holding that the intrusion constitutes
    the tort. Publication is not one of the elements.
    \item Dissent: information obtained through these means would not be 
    admissible as evidence in court, but we allow it for news media. ``There 
    is an anomaly lurking in this situation: the news media regard themselves 
    as quasi-public institutions yet they deman immunity from the restraints 
    which they vigorously demand be placed on government.''\footnote{Casebook 
    p. 746.}
\end{enumerate}

\subsubsection{\emph{Dietemann v. Time, Inc.}}

\begin{enumerate}
    \item Employees of Time collaborated with the District Attorney's office 
    to fraudulently gain access to Dietemann's home. Dietemann was suspect of 
    practicing medicine without a license. They secretly recorded 
    conversations and events. When Dietemann was arrested, the reporters took 
    pictures, which Dietemann consented to only because he thought the police 
    officers required it. Time published the material in an article titled 
    ``Crackdown on Quackery.''
    \item The district court held that the pictures taken without Dietemann's 
    consent inside his home constituted an invasion of privacy.
    \item The Ninth Circuit affirmed, holding that Dietemann had an 
    expectation of privacy within his home. An opposite holding would chill 
    candid speech.
    \item On the free speech question: ``The First Amendment is not a license 
    to trespass, to steal, or to intrude by electronic means into the 
    precincts of another's home or office. It does not become such a license 
    simply because the person subjected to the intrusion is reasonably 
    suspected of committing a crime.''\footnote{Casebook p. 752.}
\end{enumerate}

\subsection{Appropriation of Name or Likeness}

\begin{enumerate}
    \item Defendants are liable for unauthorized appropriation of the name or 
    likeness for their own use.
    \item There is a fuzzy line between tortious appropriation and legitimate 
    use.
    \item Aspects of identity beyond name and likeness can be included (e.g., 
    \emph{Here's Johnny Portable Toilets}\footnote{Casebook p. 758.}).
    \item Celebrities can also sue under a right to publicity tort (e.g., 
    \emph{Sagan v. Apple Computer}).
\end{enumerate}

\subsubsection{Newsworthiness: \emph{Neff v. Time, Inc.}}

A photograph taken in public with consent can be published if it is 
newsworthy.

\begin{enumerate}
    \item A Sports Illustrated photographer took a picture of Neff, with 
    consent, at a Steelers game with his fly down. The magazine used it in a 
    feature on Steelers fans with the caption ``a strange kind of love.''
    \item Neff sued for (1) appropriation of name or likeness and (2) 
    publicity given to private life. 
    \item The court held for the defendant because the photo was 
    ``newsworthy.''
\end{enumerate}

\subsection{Publicity of Private Life}

\begin{enumerate}
    \item \textbf{Publicizing private details} is tortious if it would be highly 
    offensive to a reasonable person and is not of legitimate concern to the 
    public. The facts must actually be private (see \emph{Sipple} below).
\end{enumerate}

\subsubsection{No Cause of Action if Material is Already Public: \emph{Sipple 
v. Chronicle Publ'g Co.}}

A plaintiff has no action for being outed as gay in a newspaper if he is 
already out in public in other contexts.

\begin{enumerate}
    \item Sipple prevented Sara Jane Moore from shooting President Ford. The 
    San Francisco Chronicle published a piece on Sipple mentioning his role in 
    the gay community.
    \item Sipple sued on the argument that although his gay identity was known 
    locally and in some community, he had not told his family, and he suffered 
    emotional distress when they found out through the news media.
    \item The trial court granted summary judgment for the Chronicle.
    \item The appellate court affirmed, finding that the facts were newsworthy 
    and already well known publicly to many.
\end{enumerate}

\subsection{False Light}

\begin{enumerate}
    \item ``False light is established where the defendant publicizes false, 
    objectionable information about the plaintiff.'' The publicity must be 
    highly offensive to a reasonable person and there must be proof of 
    \emph{New York Times} malice.\footnote{Casebook p. 769.}
    \item False light is distinct from defamation in a few ways:
    \begin{enumerate}
        \item False light requires \textbf{\emph{publicity}} (communication to 
        the general public), while defamation only requires \emph{publication} 
        (communication to one person other than the victim).
        \item \textbf{Both private and public figures} must prove \emph{New 
        York Times} malice.
        \item Defamation requires harm to the victim's reputation. False 
        light only requires \textbf{offensiveness}.
    \end{enumerate}
\end{enumerate}

\subsubsection{Distinguishing Common Law and \emph{New York Times} Malice: 
\emph{Cantrell v. Forest City Publ'g Co.}}

Plaintiffs claiming false light must prove \emph{New York Times} malice but 
not common law malice.

\begin{enumerate}
    \item Margaret Cantrell's husband was killed in a bridge collapse. Some 
    months later, the Plain Dealer newspaper published a piece on the Cantrell 
    family's poverty after the disaster, which included a number of 
    falsehoods.
    \item In \emph{Time, Inc. v. Hill}, the Court held that newsworthy people 
    have a right to recovery when material and substantial falsification 
    occurred.\footnote{Casebook p. 767.}
    \item In \emph{New York Times v. Sullivan}, the Court held that public 
    officials can recover with proof that the defendant published the 
    false claim ``with knowledge of its falsity or in reckless disregard of 
    the truth'' (i.e., ``actual malice'').\footnote{Casebook p. 767.}
    \item The trial court struck Cantrell's demand for punitive damages, 
    finding that the defendant had not acted with malice. The appellate court 
    interpreted this finding to mean that there was no \emph{New York Times} 
    malice and therefore that the trial court should have directed a verdict 
    for the defendants.
    \item The Supreme Court held that the appellate court confused common law 
    malice with \emph{New York Times} malice. It held that there was 
    sufficient evidence for a jury to find the defendant acted with \emph{New 
    York Times} malice, even if it lacked common law malice.
\end{enumerate}

\subsection{IIED and Public Figures: \emph{Hustler Magazine v. Falwell}}

To claim intentional infliction of emotional distress from published material, 
public figures and officials must also show \emph{New York Times} malice.

\begin{enumerate}
    \item \emph{Hustler} published parody of a Campari ad campaign featuring 
    Jerry Falwell with the caption ``Jerry Falwell talks about his first 
    time'' and other jabs. The ad included a disclaimer: ``ad parody---not to 
    be taken seriously.''
    \item ``...public figures as well as public officials will be subject to 
    `vehement, caustic, and sometimes unpleasantly sharp 
    attacks.''\footnote{Casebook p. 57.}
    \item Falwell and the appellate court take the view that if an outrageous 
    utterance intended to cause emotional distress did in fact cause distress, 
    ``it is of no constitutional import whether the statement was a fact or an 
    opinion, or whether it was true or false.''\footnote{Casebook p. 58.}
    \item The Supreme Court, Justice Rehnquist: ``We conclude that public 
    figures and public officials may not recover for the tort of intentional 
    infliction of emotional distress by reason of publications such as the one 
    here at issue without showing in addition that the publication contains a 
    false statement of fact which was made with `actual malice,' i.e., with 
    the knowledge that the statement was false or with reckless disregard as 
    to whether or not it was true.''\footnote{Casebook p. 60.}
\end{enumerate}
