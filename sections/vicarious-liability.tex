\section{Vicarious Liability}

\begin{enumerate}
    \item Vicarious liability is the doctrine that holds one party liable
    because of that party's relationship to a tortfeasor.
    \item The most common form of vicarious liability is respondeat superior,
    which hold an employer liable for its employees' torts.
    \item You cannot insure for punitive damages.
\end{enumerate}

\subsection{Respondeat Superior: \emph{Rodgers v. Kemper Construction Co.}}

\begin{enumerate}
    \item Respondeat superior: an employer is liable for an employee's torts 
    committed within the scope of employment.
    \item Kemper employees frequently drank after their shifts. Herd and O'Brien 
    finished their shifts, drank a few beers, walked across the job site, and 
    asked Rodgers for a ride on the bulldozer. They beat him up when he refused. 
    Rodgers later asked Kelley to help find out his assailants' identities. As 
    Rodgers wrote down O'Brien's license plate number, Herd, O'Brien, and a 
    third, Dieffenbauch, attacked Rodgers and Kelley, causing serious injury to 
    both.
    \item The trial court found Kemper liable for the injuries under respondeat 
    superior.
    \item On appeal, Kemper argued that it could not be held liable under 
    respondeat superior because (1) the assault occurred after O'Brien and Herd 
    had finished their shift and (2) the assault was based on personal malice.
    \item The appellate court rejected both of Kemper's arguments on the grounds 
    that the injuries resulted from ``a dispute arising out of the 
    employment.''\footnote{Casebook p. 636.}
\end{enumerate}

\subsection{Going-and-Coming Rule: \emph{Caldwell v. A.R.B., Inc.}}

\begin{enumerate}
    \item Generally, there is no employer liability for an employee's
    actions while commuting.
    \item A.R.B. workers at a Shell Oil plant were sent home because of bad 
    weather. Brandon offered to give Richardson a ride home. On the way, Brandon 
    collided with Caldwell. Caldwell sued Brandon and A.R.B., alleging that 
    Brandon was acting within the scope of employment. A.R.B. filed a motion for 
    summary judgment on the grounds that Brandon was outside of the scope of 
    employment, which the trial court granted.
    \item Although A.R.B. compensated its employees for travel expenses, the 
    appellate court held that Brandon was outside the scope of employment.
\end{enumerate}

\subsection{Independent Contractors: \emph{Mavrikidis v. Petullo}}

\begin{enumerate}
    \item Gerald Petullo was driving a dump truck full of hot asphalt when he 
    collided with Mavrikidis. Petullo and his father had been working on 
    renovations of the Clar Pine service station, which Karl Pascarello owned,
    \item Malvrikidis argued that Petullo was an employee of Pascarello, but the 
    court held that he was an independent contractor. Restatement (Second) of 
    Agency lists several factors for determining whether an actor is an 
    employee.\footnote{Casebook p. 647.} Applying these factors, the court found 
    that Petullo was not an employee.
    \item The court listed three exceptions to independent contractor 
    non-liability: (1) when the principal retains control of the work, (2) when 
    the principal hires an incompetent contractor, and (3) when the work is 
    inherently dangerous. None of these exceptions applied to Petullo.
    \item Pascarello was not liable.
\end{enumerate}

\subsection{Vicarious Liability for Children: \emph{Wells v. Hickman}}

\begin{enumerate}
    \item L.H. (15) beat D.E. (12) to death. D.E.'s mother (Wells) filed a 
    wrongful death action against L.H.'s mother (Hickman) and grandparents.
    \item An Indiana statute held parents strictly liable for their children's 
    knowing, intentional, or reckless torts for damages up to \$3,000. The trial 
    court granted summary judgment for Hickman.
    \item The appellate court here reasoned that there are four common law 
    exceptions to the rule that a parent is not liable for a child's torts: (1) 
    when the parent entrusts the child with an instrumentality that may pose 
    danger to others, (2) where the child acts as the parents' agent, (3) where 
    the parent consents, and (4) where the parent fails to exercise control when 
    the parent knows or should know that injury is possible.
    \item Wells argued that the Hickman's actions fell under the fourth 
    exception and that the statute did not limit recovery. The appellate court 
    agreed.
\end{enumerate}
