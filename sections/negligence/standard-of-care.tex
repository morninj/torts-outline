\subsection{Standard of Conduct}

\subsubsection{Standard of Care during Emergencies: \emph{Cordas v. Peerless 
Transp. Co.}}

In emergencies, people are held to a lower standard of care. Or, behavior that 
would otherwise be unreasonable is allowed in emergencies.

\begin{enumerate}
    \item A man was mugged at gunpoint by two other men in New York City. He 
    chased after them. One of the muggers jumped into a taxi, held the driver 
    at gunpoint, and told him to drive. While the cab was in motion, the 
    driver jumped out, and a few seconds later, so did the hijacker. The cab 
    crashed into a sidewalk and injured the defendants.
    \item The trial court held that the driver was not negligent because he 
    acted as a reasonable person would act under similar circumstances.
    \item Courts are divided on the question of whether juries should receive 
    special instructions regarding negligence claims in emergency 
    circumstances. On the one hand, it is redundant to reiterate that a 
    defendant must be held to the standard of what a reasonable person would 
    do in a similar emergency situation. Others claim it helps clarify the 
    standard.
    \item \textbf{Conditional privilege}: choose the lesser of two harms.
\end{enumerate}

\subsubsection{Mental Illness: \emph{Breunig v. American Family Insurance Company}}

The majority view, including in California, holds that insanity is not a 
defense to negligence.

\begin{enumerate}
    \item A schizophrenic woman had a psychotic episode while driving her car. 
    The question was whether she had foreknowledge of her susceptibility to 
    such attacks.
    \item The general rule is that \textbf{insanity or another mental deficiency does 
    not limit liability for negligence}. The court here noted that the rule may 
    be too harsh because it excludes the insanity defense when a driver is suddenly 
    overcome without warning.
    \item The Supreme Court agreed with the lower courts that the defendant 
    did have the necessary foreknowledge, and held for the plaintiff.
    \item There are \textbf{two frameworks for assessing liability from the sudden 
    onset of mental illness}:
    \begin{enumerate}
        \item Fairness: it's not fair to punish someone who could not have 
        avoided having a seizure.
        \item Loss distribution: if someone has to bear the cost of repairing 
        the harm, it should be the perpetrator, not the victim.
    \end{enumerate}
    \item One view: insanity constitutes a defense if there was no warning.
    \item Majority view (including CA): insanity does not create any defense 
    as to compensatory damages. Physical ailments, however, are taken into 
    account.
\end{enumerate}

\subsubsection{Child Standard and Adult Activity: \emph{Neumann v. Shlansky}}

Children are held to the standard of a \textbf{reasonable person of like age, 
intelligence, and experience under the circumstances} unless they are engaging 
in an adult (or inherently dangerous activity).

\begin{enumerate}
    \item An eleven-year-old hit a golf ball that struck the defendant in the 
    knee, causing serious injury.
    \item Generally, children are held to the standard of a \textbf{reasonable 
    person of like age, intelligence, and experience under the circumstances.} 
    In this case, however, the child was engaging in an ``adult activity,'' 
    and therefore the court held him to the adult reasonable person standard.
    \item Some states are moving from ``adult activity'' to ``inherently 
    dangerous activity.''
\end{enumerate}

\subsubsection{Professional Standards: \emph{Melville v. Southward}}

In many professions (e.g., medicine, engineering, accounting, and a few 
others) the ``competent professional'' standard replaces the ``reasonable 
person'' standard.

\begin{enumerate}
    \item The defendant, a podiatrist, operated on the plaintiff's foot. The 
    plaintiff sued for malpractice, and introduced the testimony of an 
    orthopedist, who questioned the necessity and sanitation of the operation.  
    \item The question before the court was whether the orthopedist, a 
    practitioner from a different school of medicine, should have been allowed 
    to testify about the standard of care in podiatry.
    \item The trial court allowed the orthopedist to testify.
    \item The Supreme Court of Colorado here agreed with the appellate court 
    that the testimony should not have been allowed because it was ``nothing 
    more than an expression of opinion that that the general practice of 
    podiatry did not meet the standard of care observed by an orthopedic 
    surgeon.'' Reversed.
    \item There is disagreement about whether doctors in rural areas should be 
    held to different standards than urban doctors.
    \item Medical specialists in the same geographic region are often 
    reluctant to testify against each other---a ``conspiracy of silence.''
    \item In a limited range of cases, a jury of laypeople can determine 
    whether a practice met an acceptable standard of care.
\end{enumerate}

\subsubsection{Informed Consent: \emph{Cobbs v. Grant}}

Doctors are required to obtain \textbf{informed consent} from patients.  
Failure to obtain informed consent can expose a physician to negligence 
liability.

\begin{enumerate}
    \item The plaintiff here sued a doctor who operated on a stomach ulcer 
    but did not discuss the surgery's inherent risks. Complications developed, 
    another operation was required, more complications developed, and so on.
    \item The plaintiff argued that (1) the doctor acted negligently in the 
    performance of the surgery (on which the jury found in favor of the 
    plaintiff) and (2) that the doctor failed to obtain informed consent.
    \item The Supreme Court of California here noted that courts are divided 
    as to whether this type of tort should be deemed a \textbf{battery or 
    negligence}. The court aligned itself with a ``majority trend'' that 
    advocates reserving battery for cases where a doctor performs an operation 
    without the patient's consent.
    \item Generally, physicians are required to tell 
    patients about major risks (but not every minor risk) and obtain the 
    patient's consent.
    \item In this case, the court found that there was not enough 
    evidence to show that the doctor acted negligently. Reversed.
    \item Failure to obtain informed consent can expose a physician to 
    negligence liability. Unless the physician misrepresents the entire 
    procedure, most courts will not characterize the behavior as intentional 
    battery.
\end{enumerate}
