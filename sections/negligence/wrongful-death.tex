\subsection{Wrongful Death and Survival Actions}

\begin{enumerate}
    \item At common law, there was no recovery for wrongful death or survival 
    actions.
    \item Certain named categories of relatives can recover for 
    \textbf{wrongful death}. In all but a few jurisdictions, \textbf{damages 
    are limited to pecuniary losses, not including pain and suffering}.  
    However, \textbf{in California, survivors can collect for monetary value 
    of loss of companionship}---a way of softening the effect of not allowing 
    pain and suffering. A small minority of states also allow recovery for 
    grief.
    \item Only certain close relatives can pursue wrongful death actions. See 
    C.C.P. \S\ 377.60 below. Adopted children generally cannot recover for the 
    deaths of biological parents. Courts vary on whether parents can recover 
    for the wrongful death of an unborn child.
    \item \textbf{Survival actions} are the victim's tort claims that survive 
    the victim. The victim's heirs can pursue those claims in court. Most 
    jurisdictions, including California, \textbf{only allow survival of claims 
    for economic losses, not pain and suffering}.
    \item Wrongful death \end{enumerate}

\subsubsection{C.C.P. \S\ 377: Wrongful Death and Survival Actions in 
California}

\begin{enumerate}
    \item \textbf{Survival Actions} (\S\ 377.20): (a) A cause of action for or 
    against a person survives that person's death. (b) This section applies 
    even if the loss occurs simultaneously with or after the person's death.  
    (For example, if a driver negligently causes a car accident and dies, the 
    other driver can recover from the negligent driver's estate.)
    \item \textbf{Damages in Survival Actions} (\S\ 377.34): recovery is 
    limited to loss incurred before the victim's death.
    \item \textbf{Wrongful Death} (\S\ 377.60(a)--(c)): These people can bring 
    wrongful death actions:
    \begin{itemize}
        \item (a) The decedent's spouse, domestic partner, children, 
        grandchildren, or whoever would be entitled to the decedent's 
        property.
        \item (b) Other dependents of the decedent: putative spouse, children 
        of the putative spouse, stepchildren, or parents.
        \item (c) A minor if the minor resided with the decedent for more than 
        180 days prior to the death and relied on the decedent for half or 
        more of the minor's support.
    \end{itemize}
    \item \textbf{Wrongful Death Awards} (\S\ 377.61): Any ``damages may be 
    awarded that, under the circumstances of the case, may be just,'' 
    excluding survival actions.
\end{enumerate}

\subsubsection{Parental Actions for Wrongful Deaths of Children: \emph{Gary v.  
Schwartz}}

A child's potential contribution to the family weighs heavily in calculating 
damage awards.

\begin{enumerate}
    \item The defendant's negligent driving killed the plaintiff's 
    sixteen-year-old son. The jury returned \$100,510.40 in damages for the 
    mother.
    \item The defendant moved to set aside the verdict as excessive.
    \item The court noted that current wrongful death law warranted the 
    application of a formula in calculating damages: ``probable wages of the 
    child less cost of upkeep until the infant would have reached 21 years of 
    age.''\footnote{Casebook p. 345.} The court noted that this formula would 
    almost always result in a negative figure. The jury instructions prevented 
    awarding damages for emotional factors. However, the court found that the 
    damages were not excessive because the son ``in all likelihood...would 
    have faithfully borne the burden of caring for his mother and in aiding 
    his younger brother, if necessary, upon the completion of his 
    education.''\footnote{Casebook p. 348.}
\end{enumerate}

\subsubsection{Less Valuable Children: \emph{Selders v. Armentrout}}

There may be little recovery for children to contribute little to the family.

\begin{enumerate}
    \item The plaintiffs' three children were killed. The jury awarded damages 
    of \$1,500 per child, which covered all pecuniary loss.
    \item The appellate court found that all the children had left home when 
    they could support themselves and ``had made no contribution of earnings 
    other than to their own support.''\footnote{Casebook p. 349.} The court 
    held that damages in wrongful death cases cannot be computed by formula 
    but should be left to a jury, and in this case the jury could have 
    reasonably concluded that the loss to the parents (``including the value 
    of society and companionship'') was small.
\end{enumerate}

\subsubsection{Extremely Valuable Children: \emph{Compania Dominicana de 
Aviacion v. Knapp}}

Damages can also be high. The jury has broad discretion.

\begin{enumerate}
    \item The defendants' plane crashed into an auto body shop, killing two of 
    the plaintiffs' three sons. For one of the sons, the jury awarded \$1.8 
    million in damages.
    \item The appellate court held that the standard of review does not allow 
    an appellate court to alter a jury verdict simply because it disagrees 
    with it. In this case, the court found that the damages (which included 
    loss of services \emph{and} pain and suffering) was large but appropriate.
\end{enumerate}

