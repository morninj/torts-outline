\subsection{Limitations on Duty}

\begin{enumerate}
    \item There is generally no duty to affirmatively act, with a few 
    exceptions:
    \begin{enumerate}
        \item One who causes injury may have a duty to rescue.
        \item The relationship between plaintiff and defendant may create a duty: 
        common carrier, innkeeper, land owners, one who is required or 
        voluntarily takes custody of another to deprive the other of normal 
        opportunities for protection.\footnote{Casebook pp. 288-89.}
        \item Beginning an undertaking that places the victim in a position 
        that makes them less likely to be rescued can lead to liability.
        \item Good Samaritan statutes protect from liability.
    \end{enumerate}
\end{enumerate}

\subsubsection{Status and Duty: \emph{L. S. Ayres \& Co. v. Hicks}}

Legal status can create a duty to act.

\begin{enumerate}
    \item Hicks, a six-year-old boy, injured his fingers in the escalator at a 
    department store.
    \item There is generally no legal duty to act. However, special 
    relationships---master-servant, inviter-invitee, etc.---can give rise to 
    legal duties.
    \item The court held that in this case the injury itself was unforeseeable 
    and therefore there is no liability. However, the defendant's status as an 
    inviter established a duty to rescue. The plaintiff was entitled to 
    recover to aggravation of his injuries (but not for the initial injury).
\end{enumerate}

\subsubsection{Voluntary Interventions: \emph{Miller v. Arnal Corp.}}

There is generally no duty to act, but voluntary undertakings can create 
liability if the person giving aid puts the victim in a worse position.

\begin{enumerate}
    \item Miller was caught in a snow storm on a mountain. The ski patrol 
    planned to rescue him, but the manager refused to operate the lift because 
    of the danger from the weather. The county sheriff's rescue party reached 
    Miller several hours later after he had suffered severe hypothermia.
    \item Miller argued that the attempted rescue put him in a worse position.
    \item The court held that Miller did not rely on the ski patrol's planned 
    rescue efforts, nor did the patrol's plan delay the sheriff's rescue.
\end{enumerate}

\subsubsection{Parental Duty: \emph{Wells v. Hickman}}

Parents generally have no duty to control their children unless the parent 
knows or should know that harm is possible.

\begin{enumerate}
    \item Hickman's son, D.E., killed Wells' son, L.H.  Wells argued that 
    Hickman should be liable for failing to control her son.
    \item The court held that ``a duty attaches when there has been a failure 
    to control [the child] and the parent knows or should have known that 
    injury to another was reasonably foreseeable.''\footnote{Casebook p. 300.}
    \item The court held that Hickman did not have a duty to exercise control 
    over L.H. because the harm to D.E. was not foreseeable.
\end{enumerate}

\subsubsection{Therapists and Potential Victims: \emph{Tarasoff v. The Regents 
of the University of California}}

The relationship between the defendant and a third party can create a duty to 
a stranger.

\begin{enumerate}
    \item Moore was a Berkeley psychologist. Poddar, his patient, confided his 
    intention to kill Tarasoff. At Moore's request, the police briefly 
    detained Poddar but released him soon after. Two months later, he did kill 
    Tarasoff.
    \item The plaintiffs argued that Tarasoff's death ``proximately resulted 
    from defendants' negligent failure to warn Tatiana [Tarasoff] or others 
    likely to appraise her of her danger.''\footnote{Casebook p. 306.}
    \item The court held that duty (and hence liability) only exists when 
    there is a special relationship between the defendant and the dangerous 
    person or the potential victim. Therapists owe a duty not just to their 
    patients but also to potential victims of their potential actions.
    \item ``Most therapists are instructed to give so-called `\emph{Tarasoff} 
    warnings' in appropriate cases.''
\end{enumerate}

\subsubsection{Police Duty to the Public: \emph{Davidson v. City of Westminster}}

There is no status relationship that creates a duty for police to apprehend a 
suspected felon or to warn a potential victim. ``...police officials generally 
have no duty to the public.''\footnote{Casebook p. 318 n. 1.}

\begin{enumerate}
    \item A man had been serially stabbing women in laundromats. One night, 
    Davidson was in a laundromat. Police officers were watching the laundromat 
    when a man they suspected to be the stabber entered. The officers did not 
    intervene and Davidson was stabbed.
    \item The court held that the officers did not have a duty to stop the 
    suspected assailant. The police would only have a duty to Davidson if they 
    created her peril, which they did not, so there was no duty to warn her.
\end{enumerate}
