\subsection{Limitations on Duty}

% \item See Ca9l. Civ. Code 1714.
% \item No duty to affirmatively act, with a few exceptions:
% \begin{enumerate}
%     \item One who causes injury may have a duty to rescue.
%     \item Relationship between P and D may create a duty: common carrier, 
%     land occupiers, innkeeper, parent...
%     \item Beginning an undertaking that places the victim in a position that 
%     makes them less likely to be rescued can lead to liability.
%     \item Good samaritan statutes protect from liability.
% \end{enumerate}
% 
%\subsubsection{Failure to Act: \emph{L. S. Ayres \& Co. v. Hicks}}
%
%\begin{enumerate}
%    \item todo
%\end{enumerate}
%
%\subsubsection{\emph{Miller v. Arnal Corp.}}
%
%\begin{enumerate}
%    \item todo
%\end{enumerate}
%
%\subsubsection{\emph{Wells v. Hickman}}
%
%\begin{enumerate}
%    \item todo
%\end{enumerate}
%
%\subsubsection{\emph{Tarasoff v. The Regents of the University of 
%California}}
%
%\begin{enumerate}
%    \item todo
%    \item [Levy lecture: ]Relationship between defendant and third party can 
%    create a duty to a stranger.
%\end{enumerate}
%
%\subsubsection{\emph{Davidson v. City of Westminster}}
%
%\begin{enumerate}
%    \item todo
%\end{enumerate}
%


