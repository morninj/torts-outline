\subsection{Cause in Fact}

\begin{enumerate}
    \item Plaintiff must show that the defendant's negligence was a cause in 
    fact of the harm.
    \item Traditionally: plaintiff must prove that the harm would not have 
    occurred \textbf{but for} the defendant's actions.
    \item If there are multiple causes of harm, each can be but-for causes as 
    long as the harm would not have occurred without it.
    \item If there are multiple causes of harm, but none alone is a but-for 
    cause, courts can use the \textbf{substantial factor test}. See 
    \emph{Northington} below.
    \item Substantial factor test in the Second Restatement:
    \begin{enumerate}
        \item % todo: add
    \end{enumerate}
    \item The biggest different between but-for test and substantial factor 
    test is that under the substantial factor test, it's much more likely to 
    go to a jury.
    \item \textbf{Proximate cause} removes liability when ``the connection 
    between the plaintiff's harm and defendant's liability is unforeseeable or 
    so attenuated that public policy precludes liability.''\footnote{Casebook 
    p. 206.}
    \item When two people are ``acting in concert'' (i.e., trying to do the 
    same thing), and one is the negligent actor, the court can hold both 
    parties liable.
    \item \emph{Summers v. Tice}: CA Supreme Court adopted \textbf{alternative 
    liability test}: when one of two negligent defendants probably caused a 
    harm, and it has not been shown that it is more likely than not that 
    either caused it, then each will be held jointly and severally liable for 
    the full amount of the harm.
    \begin{enumerate}
        \item Restatement Second says the test applies when there are ``two or 
        more'' defendants.
        \item Also think about the effect of Prop 51 on \emph{Summers v. 
        Tice}: not clear whether it applies to these joint tortfeasor cases or 
        not.
    \end{enumerate}
    \item \emph{Sindell}, \textbf{market share liability}: each defendant 
    shall be held liable for the proportion of the judgment represented by its 
    share of the market unless it can demonstrate that it did not manufacture 
    the product that caused the plaintiffs' injuries. This is the case in CA, 
    but not in NY.
    \item Toxic torts: what to do when there is no harm, but only enhanced 
    risk? One approach: award damages, but to a lesser amount, based on the 
    percent chance of the harm. Seond approach (in CA): will not award general 
    damages, but will award damages for medical surveillance.
\end{enumerate}

\subsubsection{\emph{East Texas Theatres, Inc. v. Rutledge}}

\begin{enumerate}
    \item At the defendant's movie theater, somebody threw a bottle from a 
    balcony which struck and injured the plaintiff. The jury found the theater 
    liable because it negligently failed to remove ``rowdy persons'' from the 
    balcony during the game, and the Texas appellate court affirmed. The Texas 
    Supreme Court clarified that proximate cause has two elements: (1) 
    cause-in-fact and (2) foreseeability. The court held that the prosecution 
    failed to show that the injuries would have occurred but for the removal 
    of the ``rowdy persons.'' It reversed the lower court's ruling and held 
    for the defendants.
\end{enumerate}

\subsubsection{\emph{Anderson v. Minneapolis, St. P. \& S. S. M. Ry. Co.}}

\begin{enumerate}
    \item A spark from a railroad started a fire in a bog on one side of the 
    defendant's property. Another unrelated fire was burning on the other 
    side. The fire from the railroad destroyed the defendant's property, and a 
    few days later it joined with the other fire to make one big fire. The 
    railroad argues that it cannot be held liable because the defendant's 
    house would have been destroyed by the other fire anyway. The trial court 
    refused to instruct the jury to follow a rule from an earlier case, 
    \emph{Cook}, which held that there is no liability when two fires jointly 
    destroy property. On this basis, the trial court found for the plaintiff. 
    The railroad requested a motion for judgment notwithstanding the verdict, 
    which was denied. On appeal, the Supreme Court of Minnesota held that the 
    trial court was correct in refusing to apply the \emph{Cook} rule and 
    found for the plaintiffs.
    \item \textbf{Substantial factor test}: If two independent fires join to 
    cause property damage, there is joint liability, even if neither alone is 
    a but-for cause. Redundant causation is not necessary.
    \item Courts split on whether to use the substantial factor test when only 
    one actor is liable. California courts do use it (and reject the but-for 
    test).
\end{enumerate}

\subsubsection{\emph{Northington v. Marin}}

\begin{enumerate}
    \item The plaintiff, a prison inmate, sued the defendant, a prison guard, 
    for circulating rumors that labeled him a snitch and caused other inmates 
    to assault him. Other guards had spread the same rumors. The trial court 
    found that although the defendant's action was not a but-for cause (since 
    the harm would have occurred without his action), his contribution to the 
    harm was nonetheless a \textbf{substantial factor}. The Tenth Circuit 
    affirmed: ``Multiple tortfeasors who concurrently cause an indivisible 
    injury are jointly and severally liable; each can be held liable for the 
    entire injury.''
\end{enumerate}

\subsubsection{\emph{Herskovitz v. Group Health Cooperative of Puget Sound}}

\begin{enumerate}
    \item The plaintiff brought the action on behalf of her husband, a 
    deceased lung cancer patient, against a doctor that negligently failed to 
    diagnose the patient's lung cancer on his first visit, proximately causing 
    his chance of survival to drop from 39 percent to 25 percent. Neither fact 
    was in dispute. The defendant argued that the plaintiff must prove that 
    the patient ``probably'' would have lived but for the negligence---that 
    is, without the doctor's negligence, the patient's chance of survival must 
    have been more than 50 percent. The trial court granted summary judgment 
    for the defendant on this argument. The Supreme Court of Washington 
    reversed, arguing that any other decision would mean a ``blanket release'' 
    for doctors' negligence any time the patient's chance of survival was less 
    than 50 percent. The court reasoned that if a defendant's acts have 
    \emph{increased the risk} of harm to the plaintiff, a jury should decide 
    whether the increased risk actually caused the harm in question.
\end{enumerate}

\subsubsection{\emph{Summers v. Tice}}

\begin{enumerate}
    \item The \emph{Summers} rule applies where there are a small number of 
    defendants, only one of them committed the harm, and we don't know which 
    one.
    \item The plaintiff and the two defendants were hunting quail. The two 
    defendants shot at a quail in the direction of the plaintiff. The 
    plaintiff suffered injuries, but it's not clear which defendant's shot was 
    the cause. The court reasons that in this case, the burden of proof shifts 
    to the defendants to determine which one of them caused the injury. If 
    they cannot, ``each defendant is liable for the whole damage whether they 
    are deemed to be acting in concert or independently.'' The lower courts 
    found the defendants liable and the Supreme Court of California affirmed.
    \item Can you hold three defendants liable under the \emph{Summers} test?
    \item Another case with joint tortfeasors, see \emph{Drabek v. Sabley} 
    above (kids throwing snowballs at cars).
\end{enumerate}

\subsubsection{\emph{Sindell v. Abbott Laboratories}}

\begin{enumerate}
    \item The plaintiff was harmed by DES, a prenatal drug intended to protect 
    against miscarriages but which turned out to pose significant danger to 
    unborn children. The plaintiff did not know which company manufactured the 
    specific drug her mother took, but since several companies manufactured 
    the drug according to the same formula, she sued them all. The companies 
    won a dismissal at trial on the grounds that the plaintiff could not 
    identify which company caused the harm.
    \item The Supreme Court of California considered four theories of 
    liability:
    \begin{enumerate}
        \item The \emph{Summers} test: this fails because there are so many 
        defendants (over 200) that it is highly unlikely that any one of them 
        caused this specific injury.
        \item The ``concert of action'' theory: if the defendants had acted in 
        concert to cause the injury, they would be equally liable. In this 
        case, there is not sufficient evidence to show that the defendants had 
        a common plan to cause harm (e.g., by conducting inadequate safety 
        tests or giving insufficient safety warnings).
        \item ``Industry-wide'' or ``enterprise'' liability: if an entire 
        industry cooperates on an element of the harm in question---e.g., by 
        delegating safety testing to a trade association---they can be held 
        jointly liable. Here, the fact that DES manufacturers shared testing 
        and promotion methods does not establish industry-wide liability, 
        because (1) there are so many manufacturers and (2) safety standards 
        are mostly regulated by the FDA.
        \item \textbf{Market share liability}---a variation of the 
        \emph{Summers} test: each manufacturer's liability and share of the 
        damages are proportionate to its market share.
    \end{enumerate}
    \item Relying on the fourth theory, the Supreme Court of California 
    reversed, allowing the plaintiff to proceed with her cause of action.
    \item Most states have not adopted market share liability.
    \item Defendants are allowed prove definitively that they did not 
    contribute to the harm (e.g., if they can show that they did not produce 
    the drug at the time).
    \item Some states require defendants to be joined so that a significant 
    share of the market is represented, and that missing market share 
    proportionally reduces the plaintiff's compensation. Usually (but not 
    always) this must be the nationwide market.\footnote{Casebook p. 229 n. 
    2.}
\end{enumerate}

\subsubsection{\emph{Ayers v. Township of Jackson}}

\begin{enumerate}
    \item A town in New Jersey was found to have caused toxic exposure by its 
    ``palpably unreasonable'' management of a landfill. Plaintiffs did not 
    develop any illnesses, but they sought to recover (1) damages for the 
    enhanced risk of future illness due to exposure and (2) regular medical 
    testing for diseases from exposure. The Supreme Court of New Jersey found 
    that he task of litigating hypothetical injuries would unreasonably strain 
    the tort system (although it suggests that the state legislature could 
    pass a remedy that allowed damages if toxic exposure caused a 
    ``statistically significant incidence of disease''). On the second claim, 
    it held for the plaintiffs.
\end{enumerate}

