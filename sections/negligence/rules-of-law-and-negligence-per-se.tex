\subsection{Rules of Law and Negligence Per Se}

\subsubsection{Rules of Law}

\begin{enumerate}
    \item Juries typically determine what constitutes reasonable conduct under 
    the circumstances. Judges, however, will sometimes establish a 
    \textbf{rule of law} for 
    what constitutes negligent conduct under particular circumstances.
    \item For instance, in \emph{Baltimore \& Ohio R.R.}, Justice Holmes 
    established a rule requiring drivers to get out of their cars and examine 
    railroad crossings (``stop, look, and listen'').
    \item Most courts do not use this approach, because it's premised on 
    repetition of fact patterns. Fact patterns are rarely identical. Judges 
    may also be bad at making these rules (e.g., Holmes).
\end{enumerate}

\subsubsection{Rule of Law for Baseball Backstops: \emph{Akins v. Glens Falls City School District}}

\begin{enumerate}
    \item A foul ball injured a spectator at a baseball game. She sued the 
    ballpark's owners, the local school district, for negligence. The trial 
    court helf in favor of the plaintiff. The appellate court reversed, 
    finding that the school district had not acted negligently, and 
    establishing a specific rule for ballpark backstops. The dissent argued 
    that such a rule ``robs the jury'' of the ability to consider important 
    circumstances and locks the law in ``a position that is certain to become 
    outdated.''
\end{enumerate}

\subsubsection{Negligence Per Se}

\begin{enumerate}
    \item Under negligence per se, \textbf{liability exists when the defendant 
    violates a statute}.
    \item In some states, a jury \textbf{must presume} negligence when a 
    statute is breached. The defendant is free to rebut. California follows 
    this rule, with a few exceptions.\footnote{Cal. Evid. C. § 669.  See 
    course reader p. 11.} \item In states that do not follow negligence per 
    se, juries are free to (but need not) \textbf{infer} that breach of 
    statute constitutes negligence---e.g., a car doesn't slow down and hits a 
    pedestrian in a crosswalk.
    \item Plaintiff can, and usually will, plead both common law negligence 
    and negligence per se.
    \item Compliance with a statute is generally not proof of due care.
    \item ``Statutory purpose doctrine'': for the statute to be relevant, the 
    harm that occurred must have been the type that the statute was intended 
    to prevent. However, statutory purpose can sometimes be unclear, and it 
    may change through time.
    \item ``Dual purpose doctrine'': a statute may have more than one narrow 
    purpose.
    \item Generally (including Levy): proof of compliance with a statute is 
    \textbf{never} proof of due care. Criminal statutes set a minimum of 
    conduct that could be below what we'd call due care. Some cases take the 
    opposing view.
    \item A federal statute may preempt what would otherwise be a state cause 
    of action. Types of preemption: explicit, conflict, and field.
\end{enumerate}

\subsubsection{\emph{Wawanesa Mutual Insurance Co. v. Matlock}}

\begin{enumerate}
    \item A minor bought cigarettes for another minor, who later dropped the 
    cigarette and caused a fire that led to significant property damage. The 
    insurer sued the first minor's father, and the trial court found for the 
    insurer. The appellate court overturned the ruling. It argued that the 
    statute in question was meant to protect against the health hazards of 
    tobacco, not the fire hazard, and therefore cannot be used to establish a 
    standard of conduct in this case.
\end{enumerate}

\subsubsection{\emph{Stachniewicz v. Mar-Cam Corporation}}

\begin{enumerate}
    \item A patron injured in bar brawl sued the bar owner. The plaintiff 
    relied on (1) an Oregon statute which prohibits giving alcohol to an 
    intoxicated person and (2) an Oregon regulation that prevents bar owners 
    from allowing disorderly conduct on their premised. The trial court found 
    for the defendant. The appellate court overturned, reasoning that (1) the 
    statute is inapplicable because the brawler was already drunk when he 
    arrived, so there is no way to tell if another drink caused the brawl, but 
    (2) the regulation was intended specifically to protect customers from 
    injury, and therefore can be an appropriate standard for negligence in 
    this case.
\end{enumerate}

\subsubsection{\emph{Gorris v. Scott}}

\begin{enumerate}
    \item Several sheep on a ship were swept overboard. The plaintiff sued the 
    shipowner, arguing that the Contagious Diseases (Animals) Act required the 
    shipowner to enclose the sheep in pens of certain dimensions, which the 
    shipowner failed to do. The court found in favor of the shipowner, 
    reasoning that the Act was intended to prevent the spread of contagious 
    diseases, not to prevent sheep from falling overboard.
    \item \textbf{Statutory purpose doctrine}: for the statute to be relevant, 
    the harm must be one of the harms the statute was meant to prevent.
\end{enumerate}

