\subsection{Emotional Distress}

\subsubsection{Mental Distress: \emph{Thing v. La Chusa}}

\begin{enumerate}
    %%%%%%%%%%%%%%%%%%%
    \item Levy lecture 10/24: % TODO
    \begin{enumerate}
        \item Common law: no recovery unless there was physical contact. No duty to bystandars.
        \item amaya [?]: Bystander in the ``zone of danger'' can be compensated
        \item \emph{Dillon}:
        \begin{enumerate}
            \item Was the plaintiff at or near the scnee?
            \item Was the distress caused by sensory and contemporaneous observance?
            \item Does P have close relationship with V?
            \item [No physical injury or manifestation requiremenet.]
        \end{enumerate}
        \item \emph{La Chusa}: court turned dillon guidelines into requirements. a ``jurisprudence of categories.''
        \item What if ther eis only a fear of cancer? Potter: disease must be more likely than not to appear for p to recover for distress. Concedes that someone who had a 30\% chance would suffer distress, but it wanted a bright line.
        \item Asbestiosis and tobacco: move very far from what the tort system was meant to deal with. Courts develop clever solutions. Potter did allow compensation for medical monitoring.
    \end{enumerate}
    %%%%%%%%%%%%%%%%%%%%%%%%

    \item The mother of a child struck by a car sued the driver for negligent infliction of emotional distress (NIED). The trial court granted a summary judgment in favor of the defendant. The Court of Appeal reversed, holding that the mother may recover. The California Supreme Court reversed the appellate court, holding that the trial court was correct in granting summary judgment.
    \item In \emph{Amaya v. Home Ice, Fuel, \& Supply Co.}, the court held that plaintiffs must have been within the ``zone of danger'' to recover NIED damages.
    \item Five years later, the court overruled \emph{Amaya} in \emph{Dillon v. Legg}. In that case, the mother of the victim may have been endangered by the defendant's conduct, but the sister was not, leading to an incongruous result from the ``zone of danger'' test. The court held that recovery should be based on the traditional tort principles of foreseeability, proximate cause, and consequential injury. The \emph{Dillon} framework considers three factors:
    \begin{enumerate}
        \item Whether the plaintiff was \textbf{located near the scene}.
        \item Whether the plaintiff's shock resulted from the emotional impact of \textbf{``contemporaneous observance''} of the incident.
        \item Whether the plaintiff and victim are \textbf{closely related}.
    \end{enumerate}
    \item Under \emph{Dillon}, the jury will decide on a case-by-case basis ``what the ordinary man under such circumstances should reasonably have foreseen.''\footnote{Casebook p. 325.}
    \item The court here argued that \emph{Dillon} created massive uncertainty. The court replaced \emph{Dillon} with a new rule for finding NIED, which requires three factors:\footnote{Casebook p. 323.}
    \begin{enumerate}
        \item Plaintiff must be \textbf{closely related} to the victim.
        \item Plaintiff must be \textbf{present at the scene} and aware that an injury has occurred.
        \item Plaintiff must suffer \textbf{emotional distress beyond that which would be anticipated in a disinterested witness}.
    \end{enumerate}
    \item The court's motivation was to ``limit liability and establish meaningful rules for application by litigants and lower courts.''\footnote{Casebook p. 329.} Policy reasons included guarding against fraudulent claims and limiting defendants' liability.\footnote{Casebook p. 324.}
    \item The dissent argued that \emph{Dillon} was meant to be a flexible test based on the basic principles of torts The court here has replaced it with an arbitrary rule. The ``policy reasons'' for replacing the foreseeability requirement are not convincing.
    \item NIED was originally limited to cases where the plaintiff was physically impacted, i.e., for emotional distress associated with personal injuries. Virtually all courts have abandoned this rule.\footnote{Casebook p. 333.}
    \item A slight majority of courts still adhere to the ``zone of danger'' test (like that in \emph{Amaya}, above).
    \item The \emph{Dillon} approach is becoming a majority. Some jurisdictions allow non-visual perception of the accident or perception of its aftermath. The ``close relationship'' requirement generally includes only blood relatives and family memembers. California recently extended it to include domestic partners. Most \emph{Dillon} jurisdictions require a physical manifestation of the emotional distress, e.g., a heart attack or stomach pains, but not crying or insomnia.\footnote{Casebook pp. 334--336.}
    % TODO: add notes 7 and 8 on pp. 336-337
\end{enumerate}

%\subsubsection{\emph{Potter v. Firestone Tire and Rubber Co.}}
%
%\begin{enumerate}
%    \item todo
%\end{enumerate}
%
%\subsubsection{C.C.P. § 377}
%
%\begin{enumerate}
%    \item todo
%\end{enumerate}
